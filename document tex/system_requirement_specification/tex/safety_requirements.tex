The Vacu-arm has to work safely in settings such as research laboratories, classrooms, and prototyping areas. This section explains the safety procedures that must be included in both the system's hardware and software to reduce user risk, avoid inadvertent operation, and meet applicable safety standards. These requirements are based on institutional safety regulations, industry standards, and team risk assessments completed during system development.

\subsection{Laboratory equipment lockout/tagout (LOTO) procedures}
\subsubsection{Description}
The Vacu-arm must contain a Lockout/Tagout (LOTO) protocol to ensure that the system may be securely turned off and deactivated during maintenance or troubleshooting.
\subsubsection{Source}
Industry safety protocols; lab safety manual recommendations
\subsubsection{Constraints}
LOTO must be possible without causing damage to sensitive components or necessitating the use of specialized tools.
\subsubsection{Standards}
Occupational Safety and Health Standards 1910.147 - The control of hazardous energy (lockout/tagout).
\subsubsection{Priority}
Critical

\subsection{National Electric Code (NEC) wiring compliance}
\subsubsection{Description}
The Vacu-arm must follow National Electric Code (NEC) criteria to ensure safe and dependable wiring throughout the system. Furthermore, electronic components must be protected against electrostatic discharge (ESD) during handling, operation, and maintenance to avoid damage or failure.
\subsubsection{Source}
CSE Senior Design laboratory policy
\subsubsection{Constraints}
Compliance should not interfere with system functionality or modularity. Wiring must be clearly labeled, inexpensive, and simple to install. High-voltage sources, as specified by NFPA 70, should be avoided whenever possible.
\subsubsection{Standards}
NFPA 70
\subsubsection{Priority}
Critical

\subsection{RIA robotic manipulator safety standards}
\subsubsection{Description}
An emergency stop button must be built into the Vacu-arm system to immediately stop all motion and vacuum power in the event of a malfunction or user hazard.
\subsubsection{Source}
CSE Senior Design laboratory policy, Team risk analysis, Mechanical safety norms
\subsubsection{Constraints}
Must be easily accessible, clearly labeled, and mechanically isolated from other controls
\subsubsection{Standards}
ANSI/RIA R15.06-2012 American National Standard for Industrial Robots and Robot Systems, RIA TR15.606-2016 Collaborative Robots
\subsubsection{Priority}
Critical
