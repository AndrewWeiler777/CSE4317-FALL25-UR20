%%% LaTeX Template: Article/Thesis/etc. with colored headings and special fonts
%%%
%%% Source: http://www.howtotex.com/
%%% Feel free to distribute this template, but please keep to referal to http://www.howtotex.com/ here.
%%% February 2011
%%%
%%% Modified May 2018 by CDM
%%% Modified/simplified by CTC 2018 - 2025

%%%  Preamble
\documentclass[11pt,letterpaper]{article}
\usepackage[margin=1.0in]{geometry}
\usepackage[T1]{fontenc}
\usepackage[bitstream-charter]{mathdesign}
\usepackage[latin1]{inputenc}					
\usepackage{amsmath}						
\usepackage{xcolor}
\usepackage{cite}
\usepackage{hyphenat}
\usepackage{graphicx}
\usepackage{float}
\usepackage{subfigure}
\usepackage{sectsty}
\usepackage[compact]{titlesec} 
\usepackage[tablegrid]{vhistory}
\allsectionsfont{\color{accentcolor}\scshape\selectfont}

%%% Definitions
%%%%%%%%%%%%%%%%%%%%%%%%%%%%%%%%%%%%%%%%%%%%%%%%%%
% Change me to fit your team/semester information
\definecolor{accentcolor}{rgb}{0.0,0.0,0.5} 
\newcommand{\teamname}{Team Name}
\newcommand{\productname}{Product Name}
\newcommand{\coursename}{CSE 4316: Senior Design I}
\newcommand{\semester}{Spring 2025}
\newcommand{\docname}{System Requirements Specification}
\newcommand{\department}{Department of Computer Science \& Engineering}
\newcommand{\university}{The University of Texas at Arlington}
\newcommand{\authors}{Andrew Weiler\\ Andrew Whitmill \\ Jonathan Camarce \\ Samriddha Kharel \\ Seezayn Bishwokarma }

%%% Headers and footers
\usepackage{fancyhdr}
	\pagestyle{fancy}						% Enabling the custom headers/footers
\usepackage{lastpage}	
	% Header (empty)
	\lhead{}
	\chead{}
	\rhead{}
	% Footer
	\lfoot{\footnotesize \teamname \ - \semester}
	\cfoot{}
	\rfoot{\footnotesize page \thepage\ of \pageref{LastPage}}	% "Page 1 of 2"
	\renewcommand{\headrulewidth}{0.0pt}
	\renewcommand{\footrulewidth}{0.4pt}

%%% Change the abstract environment
\usepackage[runin]{abstract}			% runin option for a run-in title
%\setlength\absleftindent{30pt}			% left margin
%\setlength\absrightindent{30pt}		% right margin
\abslabeldelim{\quad}	
\setlength{\abstitleskip}{-10pt}
\renewcommand{\abstractname}{}
\renewcommand{\abstracttextfont}{\color{accentcolor} \small \slshape}	% slanted text

%%% Start of the document
\begin{document}

%%% Cover sheet
{\centering \huge \color{accentcolor} \sc \textbf{\department \\ \university} \par} % department/university info
\vspace{0.8 in} % Adjusts space between dept/school info and document/semester info
{\centering \huge \color{accentcolor} \sc \textbf{\docname \\ \coursename \\ \semester} \par} % doc/semester info
\vspace{0.5 in} % Adjusts space after document/semester info and before image.
%%%%%%%%%%%%%%%%%%%%%%%%%%%%%%%%%%%%%%%%%%%%%%%%%%%%%%%%%%%%%%%%%%%%%%%%%%%%
%   Change the graphic here. Put your image in the 'images' folder
%   and update the name from 'images/test_image' to your image name
% Begin image insertion ----------------
\begin{figure}[h!]
	\centering
   	\includegraphics[width=0.60\textwidth]{images/logosucc.png}
\end{figure}
% End image insertion ------------------
\vspace{0.35 in} % spacing after cover sheet image, before team/product info
{\centering \huge \color{accentcolor} \sc \textbf{\teamname \\ \productname} \par}
\vspace{0.35 in} % spacing before team members
{\centering \large \sc \textbf{\authors} \par}
\newpage


%\vspace{1 in}
%\centerline{January 13th, 2012}
%\newpage

%%% Revision History
%%%%%%%%%%%%%%%%%%%%%%%%%%%%%%%%%%%%%%%%%%%%%%%%%%%%%%%%%%%%%%%%%
% Each '\vhEntry' begins a new version entry, and each {} is a 
% column. Update this to reflect your version history.
\begin{versionhistory}
  	\vhEntry{0.1}{10.01.2015}{GH}{document creation}
  	\vhEntry{0.2}{10.05.2015}{AT|GH}{complete draft}
  	\vhEntry{0.3}{10.12.2015}{AT|GH}{release candidate 1}
  	\vhEntry{1.0}{10.20.2015}{AT|GH|CB}{official release}
  	\vhEntry{1.1}{10.31.2015}{AL}{added customer change requests}
\end{versionhistory}
\newpage

%%% Table of contents
\setcounter{tocdepth}{3}
\tableofcontents
\newpage

%%% List of figures and tables (optional)
\listoffigures
%\listoftables
\newpage

%%% Document sections
% The \section command creates a section, assigns it a number and gives 
% it the title in the braces. The \label command assigns a label to the 
% section, so you can insert a reference to it using the \ref{} command 
% See Future Requirements for examples. The \input command inserts the 
% contents of the text file indicated in the braces.
\section{Product Concept}
For our senior design project, we are building the Vacu-Arm, a robotic arm that uses suction to pick up and move objects. It will automate the process of transferring items quickly and accurately. This device is designed to reduce the need for manual handling. By doing so, it will help minimize human error and physical labor.

\subsection{Purpose and Use}
The Vacu-Arm will pick up objects using suction and automatically move them to a new location. It is designed to handle small to medium items safely and quickly. Users can set it up to perform repetitive tasks such as sorting, packing, or transferring parts. It should reduce manual lifting and improve accuracy in simple handling tasks.

\subsection{Intended Audience}
The Vacu-Arm is designed for workshops, and light industrial settings that need to automate simple picking and placing tasks. It could be used by manufacturers, warehouse operators, or anyone handling repetitive sorting or packaging work. The product is designed for users who want a tool to reduce manual labor. It can work as a standalone unit or as part of a larger automated system.

%%%%%%%%%%%%%%%%%%%%%%%%%%%%%%%%%%%%%%%%%%%%%%%%%%%%%%%%%%%%%%%%%%%%%%%%%%%%
%   Change the graphic here. Put your image in the 'images' folder
%   and update the name from 'images/test_image' to your image name
\begin{figure}[h!]
	\centering
   	\includegraphics[width=0.60\textwidth]{images/concept3.png}
    \caption{Vacuum Arm conceptual drawing} % Be sure to change the caption!
\end{figure}

\newpage
\section{Product Description}
The Vacu-Arm is a robotic arm that uses suction to pick up and move objects, making it ideal for repetitive tasks in workplaces like factories and warehouses. It helps reduce physical strain on workers, speeds up processes, and keeps performance consistent. The arm can run automatically or be manually controlled when needed, with a built-in camera to guide its movements. A second safety camera watches the area around it and stops the arm if it detects anything in the way, keeping people and equipment safe.

\subsection{Features \& Functions}
The Vacu-Arm is built using the UR20 robotic arm, which is known for being fast, accurate, and easy to work with. It uses a vacuum-powered gripper at the end of the arm to pick up, move, and place objects. The system connects to the arm over a network using an IP address, which lets it send and receive commands remotely. The Vacu-Arm can run tasks automatically or be controlled manually when needed, giving users flexibility for different situations.
To handle visual input, the system uses a separate microcontroller. This microcontroller processes data from two camera systems. The first camera is mounted near the gripper and helps the Vacu-Arm see and accurately move objects. The second camera is part of a safety system that watches the area around the robot. If a person or object gets too close, the microcontroller immediately stops the arm from moving to prevent accidents.
The Vacu-Arm setup includes the UR20 robot arm, the vacuum gripper, the two cameras, and the external microcontroller, all connected to each other. A control panel or screen is also included for the user to set up or adjust the system.

\subsection{External Inputs \& Outputs}
This subsection describes the key data exchanged between the Vacu-Arm system and external users or systems. The user can control the machine in two ways: by setting an automated task that specifies a pickup point and a destination point, or by manually controlling the arm using a camera system that provides visual feedback for precise movement and object handling. The table below summarizes the critical data elements involved in these processes, along with their descriptions and roles as inputs or outputs. Refer to Table 2

\begin{table}[H]
\centering
\caption{Critical External Data Flows}
\begin{tabular}{|l|p{7cm}|p{5cm}|}
\hline
\textbf{Data Name} & \textbf{Description} & \textbf{Use (Input/Output)} \\
\hline
User Command Input & Instructions from operator via control panel or interface, including task setup (pickup and destination points) or manual control commands via camera. & Input from end user \\
\hline
Task Configuration Data & Automated task parameters such as pickup location, destination location, and timing details. & Input from external system or user \\
\hline
Network Control Signal & Command messages sent over IP to control the UR20 robotic arm (e.g., move to position, activate gripper). & Output to robotic arm \\
\hline
Camera Feed (Primary) & Real-time video or images from the control camera mounted near the gripper, used to guide manual movement and object handling. & Input to microcontroller \\
\hline
Camera Feed (Safety) & Continuous video data from safety camera monitoring for obstacles or humans in the workspace. & Input to microcontroller \\
\hline
Safety Interrupt Signal & Signal from the microcontroller to immediately stop the robotic arm if an obstacle or person is detected nearby. & Output to robotic arm/controller \\
\hline
Status Feedback & System updates including task progress, error codes, and overall system status sent to the user interface. & Output to end user \\
\hline
Operation Logs & Records of movements, errors, and task completions for monitoring and maintenance. & Output to external system or user \\
\hline
\end{tabular}
\label{tab:dataflows}
\end{table}



\subsection{Product Interfaces}
The Vacu-Arm provides several operational interfaces tailored to different types of users, including end-users or operators, administrators, and maintainers. The primary interface for operators is a touchscreen control panel located near the base of the robotic arm. This panel displays a simple, intuitive graphical user interface that allows operators to initiate automated tasks by specifying pickup and destination points, or switch to manual control mode. In manual mode, the interface provides a live video feed from the primary camera mounted near the gripper, enabling operators to precisely guide the arm's movement and activate the vacuum gripper in real time.

The control panel also presents status updates, such as current task progress, system health, and any active safety alerts. Operators receive immediate notifications if the safety camera system detects an obstacle or person, prompting the arm to halt. Buttons for starting, pausing, and stopping tasks are clearly displayed, along with emergency stop controls for quick intervention.

Administrators access a more advanced interface, which may be hosted on a connected computer or tablet, to configure system parameters, manage user permissions, and update task routines. This interface includes dashboards that log operation history and error reports, assisting in system optimization and troubleshooting.

Maintainers interact with the Vacu-Arm via service panels on the hardware itself, which allow physical access to components like the vacuum pump and wiring. They also use diagnostic tools accessible through the administrator interface to monitor system performance, run calibration routines, and perform software updates.

\newpage
\section{Customer Requirements}
\label{customer_recs}
This product is intended to support general-purpose industrial automation using the Universal Robots UR20 collaborative robotic arm. While no specific customer or sponsor has been identified at this stage, the intended audience includes manufacturing engineers, automation specialists, and factory floor managers seeking to integrate robotic systems into their workflows. The UR20 arm will be configured to perform a variety of tasks such as object manipulation, machine tending, and precision assembly, with an emphasis on flexibility, safety, and ease of integration. The customer requirements outlined in this section define the essential features, behaviors, and performance expectations from an end-user perspective. These include intuitive user interfaces, reliable performance under factory conditions, compatibility with standard industrial protocols, and built-in safety features to support collaborative use. These requirements reflect common industry needs and will serve as the foundation for design and validation throughout the development lifecycle.

\subsection{Requirement Name}
\subsubsection{Description}
The UR20 robotic arm will be operable through a user-facing interface that supports both voice commands and real-time hand gesture
control. This dual-mode input system will enhance accessibility and flexibility on the factory floor, allowing operators to interact with the robot without physical contact or traditional input devices.

Voice control will be facilitated through a built-in microphone array and speech recognition engine (e.g., Vosk or Whisper), configured to recognize a standardized set of industrial task commands (e.g., ''Pick up part A,'' ''Rotate 90 degrees,'' ''Stop motion''). The system will use natural language processing to interpret the intent of the command and provide audible feedback through a speaker or screen interface.

\subsubsection{Source}
User feedback and real-time control expectations
\subsubsection{Constraints}
Fail-safes must be implemented to immediately halt arm motion in case of misinterpreted voice or gesture commands
\subsubsection{Standards}
ISO 10218-1 (robot manufacturers) \\
ISO 10218-2 (system integrators and end-users)
\subsubsection{Priority}
Moderate

\subsection{Requirement Name}
\subsubsection{Description}
Hand gesture control will be implemented using a depth-sensing camera (e.g., Intel RealSense or a stereo camera system), capable of tracking the operator's hand position and movement in three dimensions. Key gestures, such as open hand for ''release,'' closed fist for ''grip,'' or pointing for directional motion, will be interpreted by a real-time computer vision model using a trained neural network or rule-based tracking.
\subsubsection{Source}
Educational use case goals; instructor feedback
\subsubsection{Constraints}
Fail-safes must be implemented to immediately halt arm motion in case of misinterpreted voice or gesture commands
\subsubsection{Standards}
ISO 10218-2 (system integrators and end-users) \\
ISO/TS 15066 (Collaborative Robot Safety)
\subsubsection{Priority}
Moderate

\newpage
\section{Packaging Requirements}
\label{packaging_recs}
This section describes how the final system will be packaged and presented to the end-user upon delivery. Packaging includes both the physical form factor of hardware components and the distribution method for software components. As this product is intended for general industrial use, it will be designed for ease of deployment, setup, and maintenance by manufacturing engineers or automation specialists. The packaging must ensure protection of sensitive components, compliance with industrial aesthetics and safety standards, and ease of installation. Where applicable, branding, labeling, and user documentation must be included. These packaging requirements are distinct from operational or functional requirements and are focused solely on the product's final physical and distributable form.

\subsection{Requirement Name}
\subsubsection{Description}
A touchscreen or monitor-like display will be physically mounted in close proximity to the UR20 robotic arm to provide real-time system feedback, control status, and manual override capabilities. This monitor will serve as the primary human-machine interface (HMI) and must be designed for both ease of visibility and industrial durability.
\subsubsection{Source}
UR20 arm manufacturing, instructor
\subsubsection{Constraints}
Operator Safety:
The mounting system must ensure the monitor remains securely fastened under vibration, movement, or accidental bumps. Loose or improperly mounted displays could fall or swing into an operator, posing injury risks.
\subsubsection{Standards}
ISO 10218-2:2011 (Robots and Robotic Devices)
\subsubsection{Priority}
Moderate
\newpage
\section{Performance Requirements}
\label{performance_recs}
Performance requirements for \textbf{The Vacu-arm} address operational speed, responsiveness, energy efficiency, and overall usability in time-constrained environments such as classrooms or labs. These requirements help ensure the system performs reliably under expected workloads and can be set up or operated efficiently. They also define expectations for motor responsiveness, system startup time, and power constraints, all of which are critical for maintaining user satisfaction and minimizing downtime.

\subsection{Movement Response Time}
\subsubsection{Description}
\textbf{The Vacu-arm} must respond to movement commands within 200 milliseconds of input, ensuring low-latency control and smooth operation during manual or automated tasks.
\subsubsection{Source}
User feedback and real-time control expectations
\subsubsection{Constraints}
Performance may vary depending on USB communication speed; all operations must remain under 250ms even under load.
\subsubsection{Standards}
ISO 9283 (Manipulating Industrial Robots - Performance Criteria) \\
ROS real-time communication guidelines (if used)
\subsubsection{Priority}
High

\subsection{Startup Time}
\subsubsection{Description}
The system must complete initialization and become operational within 10 seconds of being powered on or connected to a host computer.
\subsubsection{Source}
Usability best practices; team-defined usability targets
\subsubsection{Constraints}
Includes firmware boot-up, serial connection initialization, and readiness check.
\subsubsection{Standards}
IEEE 1471 (Software Architecture) \\
ISO/IEC 25010 (System Usability Characteristics)
\subsubsection{Priority}
Medium

\subsection{Setup Time}
\subsubsection{Description}
Initial setup (physical connection and software recognition) must take no longer than 5 minutes, assuming no prior technical knowledge from the user.
\subsubsection{Source}
Educational use case goals; instructor feedback
\subsubsection{Constraints}
Includes unpacking, connecting hardware, launching software, and verifying connection.
\subsubsection{Standards}
ISO 9241-110 (Interaction principles for usability) \\
IEEE 1063 (User documentation)
\subsubsection{Priority}
High

\newpage
\section{Safety Requirements}
\label{safety_recs}
The Vacu-arm has to work safely in settings such as research laboratories, classrooms, and prototyping areas. This section explains the safety procedures that must be included in both the system's hardware and software to reduce user risk, avoid inadvertent operation, and meet applicable safety standards. These requirements are based on institutional safety regulations, industry standards, and team risk assessments completed during system development.

\subsection{Laboratory equipment lockout/tagout (LOTO) procedures}
\subsubsection{Description}
The Vacu-arm must contain a Lockout/Tagout (LOTO) protocol to ensure that the system may be securely turned off and deactivated during maintenance or troubleshooting.
\subsubsection{Source}
Industry safety protocols; lab safety manual recommendations
\subsubsection{Constraints}
LOTO must be possible without causing damage to sensitive components or necessitating the use of specialized tools.
\subsubsection{Standards}
Occupational Safety and Health Standards 1910.147 - The control of hazardous energy (lockout/tagout).
\subsubsection{Priority}
Critical

\subsection{National Electric Code (NEC) wiring compliance}
\subsubsection{Description}
The Vacu-arm must follow National Electric Code (NEC) criteria to ensure safe and dependable wiring throughout the system. Furthermore, electronic components must be protected against electrostatic discharge (ESD) during handling, operation, and maintenance to avoid damage or failure.
\subsubsection{Source}
CSE Senior Design laboratory policy
\subsubsection{Constraints}
Compliance should not interfere with system functionality or modularity. Wiring must be clearly labeled, inexpensive, and simple to install. High-voltage sources, as specified by NFPA 70, should be avoided whenever possible.
\subsubsection{Standards}
NFPA 70
\subsubsection{Priority}
Critical

\subsection{RIA robotic manipulator safety standards}
\subsubsection{Description}
An emergency stop button must be built into the Vacu-arm system to immediately stop all motion and vacuum power in the event of a malfunction or user hazard.
\subsubsection{Source}
CSE Senior Design laboratory policy, Team risk analysis, Mechanical safety norms
\subsubsection{Constraints}
Must be easily accessible, clearly labeled, and mechanically isolated from other controls
\subsubsection{Standards}
ANSI/RIA R15.06-2012 American National Standard for Industrial Robots and Robot Systems, RIA TR15.606-2016 Collaborative Robots
\subsubsection{Priority}
Critical

\newpage
\section{Maintenance \& Support Requirements}
\label{maintenance_recs}
\textbf{The Vacu-arm} will require ongoing maintenance and support to ensure its continued operation in classroom, research, and prototyping environments. Maintenance and support responsibilities include error correction, hardware replacement, troubleshooting support, and access to necessary documentation and source code. These requirements are defined with the assumption that either internal technical staff or external customers may need to support \textbf{The Vacu-arm} "in the field." To that end, documentation, tooling, and source code accessibility are critical to sustaining system performance after initial deployment.

\subsection{Access to Source Code and Documentation}
\subsubsection{Description}
Full source code and technical documentation for \textbf{The Vacu-arm}, including build instructions, interface specifications, and system architecture, must be provided. This ensures maintainers can debug, rebuild, or enhance the system when necessary.
\subsubsection{Source}
Open-source practices, internal maintenance team requirements
\subsubsection{Constraints}
All documentation must be in human-readable format (e.g., Markdown, PDF) and stored in a version-controlled repository (e.g., GitHub or GitLab).
\subsubsection{Standards}
IEEE 828 (Configuration Management) \\
ISO/IEC 26514 (Documentation for software users)
\subsubsection{Priority}
High

\subsection{Troubleshooting and Maintenance Guides}
\subsubsection{Description}
Detailed maintenance and troubleshooting manuals must be developed for both hardware and software subsystems of \textbf{The Vacu-arm}. These guides should include diagnostics steps, error codes, and common failure resolutions.
\subsubsection{Source}
User experience expectations; internal QA policy
\subsubsection{Constraints}
Documentation must be understandable by users with limited robotics experience; visuals or diagrams should accompany textual instructions.
\subsubsection{Standards}
ISO/IEC 26513 (System and software engineering -- Requirements for testers) \\
ISO 9241-110 (User interfaces)
\subsubsection{Priority}
High

\subsection{Tooling Requirements for Maintenance}
\subsubsection{Description}
All specialized tools (hardware or software) required to maintain \textbf{The Vacu-arm} must be documented and, where possible, distributed with the product or made easily accessible. These include specific calibration tools, drivers, or interface software.
\subsubsection{Source}
Hardware team input; sprint testing observations
\subsubsection{Constraints}
Avoid reliance on proprietary tools unless necessary; prioritize open-source or low-cost maintenance solutions.
\subsubsection{Standards}
ISO/IEC 19770 (Software Asset Management) \\
General ESD (Electrostatic Discharge) handling best practices
\subsubsection{Priority}
Medium

\subsection{Support Environment Specifications}
\subsubsection{Description}
A clear specification of the software environment required to run diagnostics and maintenance tasks on \textbf{The Vacu-arm} must be provided. This includes supported OS platforms, Python versions, and necessary libraries.
\subsubsection{Source}
Software team input; platform compatibility goals
\subsubsection{Constraints}
Environment must be cross-platform (Windows, Linux, macOS) and easily reproducible with documented setup steps or environment files (e.g., \texttt{requirements.txt}).
\subsubsection{Standards}
PEP 508 (Python dependency specification) \\
Docker or virtual environment usage standards (if applicable)
\subsubsection{Priority}
Moderate

\newpage
\section{Other Requirements}
\label{other_recs}
\textbf{The Vacu-arm} is a programmable vacuum-based robotic arm system designed for precision material handling in research, educational, and prototyping environments. To ensure the system meets user expectations and performs reliably in a range of scenarios, this section outlines additional requirements not previously covered. These include configuration steps for customer setup, platform compatibility, software modularity, and architectural guidelines to support future enhancements. Meeting these requirements is essential for the successful deployment and long-term scalability of \textbf{The Vacu-arm}.

\subsection{Modular Design}
\subsubsection{Description}
\textbf{The Vacu-arm} must follow a modular software and hardware design, allowing individual components (e.g., vacuum controller, movement controller, sensor feedback) to be updated, replaced, or extended independently without requiring major rework of the entire system.
\subsubsection{Source}
Engineering best practices, team decision during Sprint 2.
\subsubsection{Constraints}
All modules must follow a defined API/communication protocol to ensure compatibility. Any changes to one module must not break the functionality of others.
\subsubsection{Standards}
IEEE 1471 (Software Architecture) \\
ISO/IEC 42010 \\
ROS modularity guidelines
\subsubsection{Priority}
High

\subsection{Extensibility for Future Enhancements}
\subsubsection{Description}
\textbf{The Vacu-arm} must be designed with extensibility in mind, allowing new features such as additional sensors, AI-based decision-making, or alternate gripper types to be integrated with minimal refactoring.
\subsubsection{Source}
Team foresight, advisor recommendation
\subsubsection{Constraints}
New features must integrate using the existing control and communication structure. Extensions must be backward-compatible.
\subsubsection{Standards}
SOLID principles (Software design) \\
IEEE 830 (Software Requirements Specification)
\subsubsection{Priority}
Medium

\subsection{Portability Across Platforms}
\subsubsection{Description}
The software controlling \textbf{The Vacu-arm} must be portable across major operating systems, including Windows, Linux, and macOS, without requiring platform-specific modifications.
\subsubsection{Source}
User requirement; educational use across diverse lab setups.
\subsubsection{Constraints}
Use only cross-platform libraries (e.g., Python standard library, PySerial for communication). Avoid OS-specific commands or tools.
\subsubsection{Standards}
POSIX compliance where applicable \\
Python PEP 8 for cross-platform scripting \\
USB/serial communication standards
\subsubsection{Priority}
Medium

\subsection{Customer Setup and Configuration}
\subsubsection{Description}
The final product must include a simple configuration guide and scripts to enable users to set up \textbf{The Vacu-arm} with minimal technical background. The system should support plug-and-play recognition when connected to a host computer.
\subsubsection{Source}
Usability and accessibility goals; feedback from target users.
\subsubsection{Constraints}
Setup process must not require manual driver installation or low-level configuration changes. Must work with default Python installation and common USB interfaces.
\subsubsection{Standards}
ISO 9241-210 (Ergonomics of Human-System Interaction) \\
USB HID standards (if applicable)
\subsubsection{Priority}
High

\newpage
\section{Future Items}
In this section, we summarize requirements that were discussed during development and assigned a \textbf{Priority 5 (Future)}. These items are considered valuable enhancements to \textbf{The Vacu-arm} system but are outside the scope of the current prototype due to constraints such as time, budget, or technical limitations. The items listed here are duplicates of entries from Sections 3 through 8.

\subsection{Advanced Voice and Gesture Control Interface}
\subsubsection{Description}
The system will support real-time voice and hand gesture control to allow for contactless operation of the Vacu-arm. Operators will be able to speak commands or use predefined gestures (e.g., closed fist to grip, open hand to release) recognized by a depth-sensing camera.
\subsubsection{Source}
User feedback; industrial control trends; educational goals
\subsubsection{Constraints}
Accurate speech and gesture recognition requires robust AI models and error handling to prevent unsafe actions.
\subsubsection{Standards}
ISO 10218-1 and ISO 10218-2 (Industrial robot safety) \\
ISO/TS 15066 (Collaborative robot safety)
\subsubsection{Priority}
Future

\subsection{Mobile Power Integration}
\subsubsection{Description}
Future versions of the Vacu-arm may include a battery pack or portable power supply to support mobile or off-grid operation.
\subsubsection{Source}
Team roadmap; mobile use case exploration
\subsubsection{Constraints}
Battery weight, safety, and power capacity must be balanced for real-world tasks.
\subsubsection{Standards}
IEC 61960 (Battery performance testing) \\
ISO 26262 (Functional safety for electrical/electronic systems)
\subsubsection{Priority}
Future

\subsection{Remote Web Interface for Monitoring}
\subsubsection{Description}
The system may include a remote-access dashboard for monitoring Vacu-arm task performance, system health, and video feeds via browser.
\subsubsection{Source}
Future system extensibility planning
\subsubsection{Constraints}
Requires secure IP-based access, user authentication, and live data streaming support.
\subsubsection{Standards}
OWASP Web Security Standards \\
ISO/IEC 27001 (Information Security Management)
\subsubsection{Priority}
Future

\newpage

%%% References
\bibliographystyle{reference/IEEEtran_custom}
\bibliography{reference/refs}{}

\end{document}