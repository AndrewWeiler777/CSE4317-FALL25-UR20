%%% LaTeX Template: Article/Thesis/etc. with colored headings and special fonts
%%%
%%% Source: http://www.howtotex.com/
%%% Feel free to distribute this template, but please keep to referal to http://www.howtotex.com/ here.
%%% February 2011
%%%
%%% Modified January 2016 by CDM
%%% Modified/simplified by CTC 2018 - 2023

%%%  Preamble
\documentclass[11pt,letterpaper]{article}
\usepackage[margin=1.0in]{geometry}
\usepackage[T1]{fontenc}
\usepackage[bitstream-charter]{mathdesign}
\usepackage[latin1]{inputenc}					
\usepackage{amsmath}						
\usepackage{xcolor}
\usepackage{cite}
\usepackage{hyphenat}
\usepackage{graphicx}
\usepackage{float}
\usepackage{subfigure}
\usepackage{sectsty}
\usepackage[compact]{titlesec} 
\usepackage[tablegrid]{vhistory}
\usepackage{pbox}
\allsectionsfont{\color{accentcolor}\scshape\selectfont}

%%% Definitions
%%%%%%%%%%%%%%%%%%%%%%%%%%%%%%%%%%%%%%%%%%%%%%%%%%
% Change me to fit your team/semester information
\definecolor{accentcolor}{rgb}{0.0,0.0,0.5} 
\newcommand{\teamname}{The Suctioneers}
\newcommand{\productname}{The Vacu-Arm}
\newcommand{\coursename}{CSE 4317: Senior Design II}
\newcommand{\semester}{Fall 2025}
\newcommand{\docname}{Architectural Design Specification}
\newcommand{\department}{Department of Computer Science \& Engineering}
\newcommand{\university}{The University of Texas at Arlington}
\newcommand{\authors}{Andrew Weiler \\ Andrew Whitmill \\ Samriddha Kharel \\ Seezayn Bishwokarma \\ Jonathan Camarce}

%%% Headers and footers
\usepackage{fancyhdr}
	\pagestyle{fancy}						% Enabling the custom headers/footers
\usepackage{lastpage}	
	% Header (empty)
	\lhead{}
	\chead{}
	\rhead{}
	% Footer
	\lfoot{\footnotesize \teamname \ - \semester}
	\cfoot{}
	\rfoot{\footnotesize page \thepage\ of \pageref{LastPage}}	% "Page 1 of 2"
	\renewcommand{\headrulewidth}{0.0pt}
	\renewcommand{\footrulewidth}{0.4pt}

%%% Change the abstract environment
\usepackage[runin]{abstract}			% runin option for a run-in title
%\setlength\absleftindent{30pt}			% left margin
%\setlength\absrightindent{30pt}		% right margin
\abslabeldelim{\quad}	
\setlength{\abstitleskip}{-10pt}
\renewcommand{\abstractname}{}
\renewcommand{\abstracttextfont}{\color{accentcolor} \small \slshape}	% slanted text

%%% Start of the document
\begin{document}

%%% Cover sheet
{\centering \huge \color{accentcolor} \sc \textbf{\department \\ \university} \par} % department/university info
\vspace{1 in}
{\centering \huge \color{accentcolor} \sc \textbf{\docname \\ \coursename \\ \semester} \par} % doc/semester info
\vspace{0.45 in} % spacing before cover sheet image
%%%%%%%%%%%%%%%%%%%%%%%%%%%%%%%%%%%%%%%%%%%%%%%%%%%%%%%%%%%%%%%%%%%%%%%%%%%%
%   Change the graphic here. Put your image in the 'images' folder
%   and update the name from 'images/test_image' to your image name
\begin{figure}[h!]
	\centering
   	\includegraphics[width=0.60\textwidth]{images/logosucc.png}
\end{figure}
\vspace{0.4 in} % spacing after cover sheet image, before team/product info
{\centering \huge \color{accentcolor} \sc \textbf{\teamname \\ \productname} \par}
\vspace{0.4 in} % spacing before team members
{\centering \large \sc \textbf{\authors} \par}
\newpage


%\vspace{1 in}
%\centerline{January 13th, 2012}
%\newpage

%%% Revision History
%%%%%%%%%%%%%%%%%%%%%%%%%%%%%%%%%%%%%%%%%%%%%%%%%%%%%%%%%%%%%%%%%
% Each '\vhEntry' begins a new version entry, and each {} is a 
% column. Update this to reflect your version history.
\begin{versionhistory}
  	\vhEntry{0.1}{8.22.2025}{CC}{document creation}
  	\vhEntry{0.2}{9.19.2025}{AW|AW|SK|JC}{complete draft}
  	\vhEntry{1.0}{11.26.2025}{AW}{edited final draft}
\end{versionhistory}
\newpage

%%% Table of contents
\setcounter{tocdepth}{2}
\tableofcontents
\newpage

%%% List of figures and tables (optional)
\listoffigures
\listoftables
\newpage

%%% Document sections
% The \section command creates a section, assigns it a number and gives it the title in the braces.
% The \input command inserts the contents of the text file indicated in the braces.
\section{Introduction}
The Vacu-Arm is a robotic arm system engineered to automate pick-and-place operations using a vacuum-based suction mechanism. Built around the Universal Robots UR20 robotic arm and controlled via a ROS-based software architecture, the Vacu-Arm is capable of safely handling objects up to 50 pounds with precision and repeatability. This system is designed to enhance efficiency in light industrial settings by reducing manual labor and human error in repetitive handling tasks such as sorting, packaging, or part transfer.

This document is built upon the System Requirements Specification (SRS) and Architectural Design Specification (ADS) to define the functional and structural design decisions. For a comprehensive understanding of the system's high-level requirements and architectural framework refer to the SRS and the ADS, which serve as the foundation for the detailed design decisions presented in this specification.
\newpage
\section{System Overview}
The overall goal of the system is to develop a functional vacuum gripping setup with sufficient components and documentation to support future modifications. The UR20 robot arm serves as the core of the Vacu-Arm system, enabling precise movement of objects in three-dimensional space. Attached to the arm is the vacuum gripper, which connects to and securely holds objects for transfer during operation. While the UR20 arm is the primary component, both subsystems rely on PolyScope-the graphical interface used to program the UR20 and manage all connected components through the power box.

%%%%%%%%%%%%%%%%%%%%%%%%%%%%%%%%%%%%%%%%%%%%%%%%%%%%%%%%%%%%%%%%%%%%%%%%%%%%%%%%%%%%%%%
% If you want to change the image, put your image in the images folder and change 
% "data_flow" to the name of your image. You can also change the caption (System 
% architecture) to something else if you want.
\begin{figure}[h!]
	\centering
 	\includegraphics[width=0.50\textwidth]{images/system_overview.png} % Change me
 \caption{System Architecture}
\end{figure}

\newpage
\section{Subsystem Definitions \& Data Flow}
%%%%%%%%%%%%%%%%%%%%%%%%%%%%%%%%%%%%%%%%%%%%%%%%%%%%%%%%%%%%%%%%%%%%%%%%%%%%
%   Change the graphic here. Put your image in the 'images' folder
%   and update the name from 'images/test_image' to your image name
\begin{figure}[h!]
	\centering
 	\includegraphics[width=0.7\textwidth]{images/subsystems_dataflow.png}
 \caption{Vacu-arm data flow diagram} % Be sure to change the caption!
\end{figure}

\newpage
\section{PLC Layer Subsystems}
\input{tex/plc_layer_subsystems.tex}
\newpage
\section{Safety Sensor Layer Subsystems}
The UR20 has a Datalogic sensor attached to the PLC that sends in signals if a person or object has entered a zone. The zones can be monitored and changed on the desktop next to the PLC using the DLsentinel application.

%%%%%%%%%%%%%%%%%%%%%%%%%%%%%%%%%%%%%%%%%%%%%%%%%%%%%%%%%%%%%%%%%%%%%%%%%%%%
%   Change the graphic here. Put your image in the 'images' folder
%   and update the name from 'images/test_image' to your image name
\begin{figure}[h!]
	\centering
 	\includegraphics[width=0.60\textwidth]{images/safety_sens.png}
 \caption{Safety sensor diagram} % Be sure to change the caption.
\end{figure}

\subsection{UR20 arm halt}
The controller will determine whether there is a person within the working vicinity of the UR20 robot and signal the PLC to halt for protection.

\subsubsection{Assumptions}
The UR20 should have collision detection and should stop at impact.

\subsubsection{Responsibilities}
To signal the PLC the presence of a person near the robot using a computer vision algorithm.

\subsubsection{Subsystem Interfaces}

\begin {table}[H]
\caption {Speed Controller Subsystem Interface} 
\begin{center}
    \begin{tabular}{ | p{1cm} | p{6cm} | p{3cm} | p{3cm} |}
    \hline
    ID & Description & Inputs & Outputs \\ \hline
    \#06 & send signal & \pbox{3cm}{presence detection} & \pbox{3cm}{presence signal}  \\ \hline
    \end{tabular}
\end{center}
\end{table}

\subsection{Computer vision algorithm}

\subsubsection{Assumptions}
A computer vision algorithm to scan for the presence of the shape of a person being near the UR20 robot.

\subsubsection{Responsibilities}
The camera will have a vantage point to be able to see all around the robot.

\subsubsection{Subsystem Interfaces}

\begin {table}[H]
\caption {Computer Vision Subsystem Interface} 
\begin{center}
    \begin{tabular}{ | p{1cm} | p{6cm} | p{3cm} | p{3cm} |}
    \hline
    ID & Description & Inputs & Outputs \\ \hline
    \#07 & presence detection & \pbox{3cm}{camera reading} & \pbox{3cm}{presence detection}  \\ \hline
    \end{tabular}
\end{center}
\end{table}
\newpage
\section{UR20 Arm Layer Subsystems}
\input{tex/arm_layer_subsystems}
\newpage
\section{Conveyor Belt Layer Subsystems}
\input{tex/conveyor_layer_subsystems.tex}
\newpage

%%% References
%\bibliographystyle{plain}
\bibliographystyle{reference/IEEEtran_custom}
\bibliography{reference/refs}{}

\end{document}