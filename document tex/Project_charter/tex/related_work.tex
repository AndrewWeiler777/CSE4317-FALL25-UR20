Automated material handling systems are a rapidly advancing area of research and development due to the growing demand for efficiency and labor reduction in industrial logistics. Several solutions currently exist for automated loading and unloading, but each have limitations that our proposed system aims to overcome.

Industrial robotic arms from companies such as \textbf{KUKA} and \textbf{ABB} offer vacuum-based gripping solutions for palletizing and depalletizing applications~\cite{kuka2022, abb2023}. These systems are highly effective in controlled environments, but they are often expensive, rigid in design, and not easily adaptable to variable object types or mixed load pallets. Moreover, they typically require complex programming and significant infrastructure investment~\cite{abb2023}.

Research has also explored soft robotics and adaptive gripping technologies. For example, researchers have developed soft suction grippers that use variable vacuum to adapt to irregular surfaces~\cite{softgripper2019}. Although promising, many of these prototypes are limited to lab environments and lack the structural robustness and power needed for larger payloads encountered in freight applications.

Academic work such as the \textit{Vacuum Gripper with Shape Adaptation}~\cite{multicellvacuum2018} has shown that multi-cell vacuum pads can conform to different surfaces. However, such systems still lack the intelligent control and real-time feedback needed for consistent reliability in unstructured truck or pallet environments.

In the enthusiast and maker community, various Arduino-based vacuum lifting arms have emerged, demonstrating low-cost potential~\cite{arduino2021}. However, these lack safety features, robust control systems, and industrial-grade suction capabilities.

Compared to these solutions, our system emphasizes cost efficiency, adaptability, and ease of deployment. It will incorporate sensors for object detection, automatically adjust suction levels, and maintain a compact form factor suitable for smaller warehouses or retrofitting onto existing logistics platforms. Commercial systems are often too costly or require dedicated setups; our design aims to be deployable in semi-structured and dynamically changing environments with minimal training and configuration.


%%%%%%%%%%%%%%%%%%%%%%%%
% To cite something, use the \cite command with the name of the bibtex entry in the curly braces.
% It will determine which reference number it is and insert that number where the \cite command is.
% e.g. \cite{Rubin2012}